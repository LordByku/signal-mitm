\documentclass[a4paper, 12pt, DIV=15]{scrartcl}

\usepackage{array}
\usepackage{booktabs}
\usepackage{enumitem}
\usepackage[T1]{fontenc}
\usepackage{graphicx}
\usepackage[hidelinks]{hyperref}
\usepackage[utf8]{inputenc}
\usepackage{pgfgantt}
\usepackage[rm]{roboto}
\usepackage{scrlayer-scrpage}
\usepackage{xcolor}

\RedeclareSectionCommand[
  beforeskip=-.5\baselineskip,
  afterskip=.25\baselineskip]{section}

\renewcommand{\familydefault}{\sfdefault}
\newcommand{\ETHlogo}[1][\textwidth]{\resizebox{#1}{!}{\includegraphics{ETHlogo}}}
\ganttset{calendar week text={W~\currentweek}}
\newcommand{\optitem}{\refstepcounter{enumi}\item[$^\star$\theenumi]}
\setlength{\parindent}{0pt}
\setlength{\parskip}{6pt}

\newcommand{\thesistype}{Bachelor's} % or "Master's"
\newcommand{\student}{Student Name}
\newcommand{\advisor}{Advisor Name}
\newcommand{\thesistitle}{Thesis Title}

\begin{document}

\noindent\includegraphics[width=.32\textwidth]{../resources/ETHlogo}
\hfill
\begin{minipage}[b]{.5\textwidth}
  \raggedleft
  \sffamily\upshape\normalsize\textbf{Network Security group}\\
  \small
  Student: \student{}\\
  Supervisor: Adrian Perrig\\
  Advisor: \advisor{}
\end{minipage}

\vspace{2cm}

\begingroup
  \centering
  \bfseries\sffamily
  \LARGE \thesistitle\\[.5em]
  \large\color{gray} \thesistype{} Thesis Proposal\par
\endgroup

\section*{Introduction}
Description of the project~\cite{SCION-Book}.

\section*{Project description}
To be determined.

\section*{Work packages}

The following tasks have to be completed in this thesis;
optional tasks are marked by a star symbol.

\begin{enumerate}[itemindent=12pt,label={\sffamily\bfseries WP\arabic*:}]
\setcounter{enumi}{-1}
\item Register thesis with study administration and define timeline.
\item Familiarize yourself with the SCION architecture, \dots, and write the ``Background'' chapter of the thesis.
\item First ``real'' task.
\optitem Optional task.
\end{enumerate}

\section*{Timeline}

The timeline will be defined during the first two weeks of the project.

% Alternative: already define the timeline now.
% In that case, a Gantt chart may be helpful.

% \begin{ganttchart}[
% hgrid,
% % vgrid,
% x unit=1.5mm,
% time slot format=isodate,
% title label font=\footnotesize,
% ]{2019-09-23}{2019-12-20}
% \gantttitlecalendar{month=shortname, week} \\
% \ganttbar{Project setup}{2019-09-23}{2019-09-30} \\
% \ganttbar{Phase 1}{2019-09-30}{2019-10-14} \\
% \end{ganttchart}

% Requirements and Contact are only required when publishing a project description
% on the website.

% \section*{Requirements}
% \begin{itemize}
% \item Foundations in networking and security
% \item Python programming is helpful but not required
% \item \dots
% \end{itemize}

% \section*{Contact}
% \advisor:

\section*{Organization}
The student and the advisor will hold weekly meetings.
During each weekly meeting, the student will briefly describe the work completed during the week and outline the work to be completed during the next week.
The advisor will, if necessary, assist the student in identifying potential future issues and discuss current issues.
Pressing complications arising between two meetings will be promptly discussed.
The advisor will assist the student towards completing any agreed-upon milestones.

\section*{Grading Scheme}

\begin{center}
\begin{tabular}{rm{.7\textwidth}}
\toprule
\bfseries Grade & \bfseries Description \\
\midrule
6.0 &  Design and implementation, as well as thesis are candidates for submission to
    an academic conference or workshop. \\ \midrule
5.5 &  Project quality significantly exceeds expectations. \\ \midrule
5.0 &  Project meets expectations. \\ \midrule
4.5 &  Project partially meets expectations and has minor deficits. \\ \midrule
4.0 &  Project meets minimum quality requirements but has major deficits and
        is clearly below expectations. \\
\bottomrule
\end{tabular}
\end{center}

\bibliographystyle{abbrv}
\begin{small}
  \bibliography{ref}
\end{small}

\end{document}
